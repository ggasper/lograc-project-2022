%\documentclass[handout]{beamer}
\documentclass{beamer}

\usetheme{Madrid}
\usecolortheme{whale}
\useinnertheme{circles} 
\useoutertheme{split}
\setbeamertemplate{blocks}[rounded][shadow=true]

\usepackage[OT2,T1]{fontenc}
\usepackage[utf8]{inputenc}
\usepackage{lmodern}
\usepackage{amsfonts,amsmath,amssymb,amsthm}
\usepackage{colortbl}
\usepackage[all]{xy}
\usepackage{tikz-cd}


%\beamertemplatenavigationsymbolsempty

\title{Algebras for a Functor}
\author[Gašper, Nejc]{Gašper Golob in Nejc Zajc}
%\institute{University of Ljubljana, Slovenia}
\date{Project presentation, 26.5.2022}



\begin{document}
%%%%%%%%%%%%%%%%%%%%%%%%%%%%%%%%%%%%%%%%
\begin{frame}
\maketitle

\end{frame}
%%%%%%%%%%%%%%%%%%%%%%%%%%%%%%%%%%%%%%%%
\begin{frame}
\frametitle{Motivation}



\end{frame}
%%%%%%%%%%%%%%%%%%%%%%%%%%%%%%%%%%%%%%%%
\begin{frame}[fragile]
\frametitle{F-Algebras}

category $C$, endofunctor $F \colon C \to C$
\\~\\
\pause


\[
\begin{tikzcd}[transform canvas={scale=1.5}]
	{FA} && A 
	\arrow["\alpha", from=1-1, to=1-3]
\end{tikzcd}
\]

\end{frame}
%%%%%%%%%%%%%%%%%%%%%%%%%%%%%%%%%%%%%%%%
\begin{frame}[fragile]
\frametitle{F-Algebras}

category $C$, endofunctor $F \colon C \to C$
\\~\\~\\~\\

\[
\begin{tikzcd}[transform canvas={scale=1.5}]
	{FA} && A \\
	\\
	{FB} && B
	\arrow["\alpha", from=1-1, to=1-3]
	\arrow["{Ff}"', from=1-1, to=3-1]
	\arrow["\beta", from=3-1, to=3-3]
	\arrow["f", from=1-3, to=3-3]
\end{tikzcd}
\]

\end{frame}
%%%%%%%%%%%%%%%%%%%%%%%%%%%%%%%%%%%%%%%%
\begin{frame}
\frametitle{Initial Objects}

Such an object $I$, that for every object $X$, \\ there exist a \textbf{unique} morphism $I \to X$.

\end{frame}
%%%%%%%%%%%%%%%%%%%%%%%%%%%%%%%%%%%%%%%%
\begin{frame}[fragile]
\frametitle{Lambek Lemma}
\begin{lemma}[Lambek]
If $I = (A, \alpha)$ is an initial algebra, then $A$ is isomorphic to $FA$ via $\alpha$.
\end{lemma}
\quad \\~\\~\\
\pause

\[
\begin{tikzcd}[transform canvas={scale=1.5}]
	FA && A \\
	\\
	{F(FA)} && FA
	\arrow["\alpha", from=1-1, to=1-3]
	\arrow["Fi"', from=1-1, to=3-1]
	\arrow["F\alpha", from=3-1, to=3-3]
	\arrow["i", from=1-3, to=3-3]
\end{tikzcd}
\]

\end{frame}
%%%%%%%%%%%%%%%%%%%%%%%%%%%%%%%%%%%%%%%%
\begin{frame}
\frametitle{Polynomial Functor}



\end{frame}
%%%%%%%%%%%%%%%%%%%%%%%%%%%%%%%%%%%%%%%%
\begin{frame}
\frametitle{Initial object in F-Algebra Category}



\end{frame}
%%%%%%%%%%%%%%%%%%%%%%%%%%%%%%%%%%%%%%%%
\begin{frame}
\frametitle{Problems in Implementation}



\end{frame}
%%%%%%%%%%%%%%%%%%%%%%%%%%%%%%%%%%%%%%%%
\begin{frame}
\frametitle{Future work}



\end{frame}
%%%%%%%%%%%%%%%%%%%%%%%%%%%%%%%%%%%%%%%%




%%%%%%%%%%%%%%%%%%%%%%%%%%%%%%%%%%%%%%%%
\end{document}
